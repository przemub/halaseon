\documentclass[12pt, a4paper]{article}

\usepackage{polski}
\usepackage[utf8]{inputenc}

\usepackage[dvipsnames]{xcolor}
\usepackage{hyperref}

\title{Hałaseon}

\date{\today}


\begin{document}

\begin{titlepage}
	\begin{flushright}
	{\large \color{Goldenrod} \emph{\textbf{Muzyka z radiowęzła to szum.}} \linebreak}
	~prezes Ryszard Szubartowski
	\end{flushright}
	\centering
	\vspace{1.5 cm}
	{\scshape Projekt na konkurs Akademia Kodowania\par}
	\vspace{1.5 cm}
	{\huge\bfseries Hałaseon\par}
	\vspace{0.1cm}
	{\scshape\Large czyli pomiar hałasu dla dobra uczniów\par}
	\vspace{2cm}
	{\Large\itshape Przemysław Buczkowski\linebreak Paweł Wieczorek\par}
	\vfill
	pod opieką\par
	mgr. inż. \textsc{Ryszarda Szubartowskiego}

	\vfill

	{\large \today\par}
\end{titlepage}

\tableofcontents

\vfill \pagebreak

\section{Wstęp}

Hałas jest jednym z~cywilizacyjnych zagrożeń czyhających na nasze zdrowie. Tak jak fast foody powodują cukrzycę~i~otyłość, a~wielogodzinne przesiadywanie przed komputerem wady postawy, tak przebywanie w miejscach o~podwyższonym natężeniu hałasu powoduje, że słyszymy znacznie gorzej niż ludzie żyjący jeszcze kilkadziesiąt lat temu.

Jakie są to miejsca? Nam w pierwszej kolejności przychodzi do głowy nasze liceum w trakcie przerwy. Setki uczniów upchanych na wąskim korytarzu i hip-hopowa muzyka z radiowęzła. Jeżeli ktoś nie lubi chodzić na koncerty muzyczne, to ciężko sobie wyobrazić, by miał gdzieś do czynienia z większym poziomem hałasu.

W szkołach województwa śląskiego przeprowadzono pomiary, z których wynikło, że średni poziom hałasu to 85 decybeli\cite{dz}, podczas gdy poziom bezpieczny to 70 decybeli\cite{wikihalas}. Trzeba tu wspomnieć, że decybel jest jednostką logarytmiczną, więc dźwięk o natężeniu około 6 decybeli większym jest odbierany przez człowieka jako dwukrotnie głośniejszy.

Nie potrzeba skomplikowanych badań, by poznać tego skutki. Wielu uczniów jak i nauczycieli skarży się na bóle głowy i uszu, migreny, nie jest w stanie się skupić i odczuwa zmęczenie przez codzienne przebywanie w tak głośnym środowisku. W dalszej części scenariusza zawarliśmy krótki przegląd wiedzy naukowej na ten temat.

Celem naszego przedsięwzięcia jest zbudowanie urządzenia, dzięki któremu będzie można skutecznie mierzyć poziom hałasu w różnych miejscach budynku szkoły. Projekt takiej sondy jest przedstawiony w niniejszym dokumencie. Po jej zbudowaniu wypróbujemy je w naszej szkole i określimy skalę problemu dla niej. Zawarliśmy też tutaj rozmaite nasze propozycje, jak wykorzystać dane, które będą zbierane przez urządzenie i~w~efekcie pomóc w~redukcji hałasu w~środowisku szkolnym i nie tylko.

\section{Szkodliwość hałasu w pigułce}

\section{Zbierane dane i ich zastosowania}

\section{Budowa sondy}

\section{Oprogramowanie sondy i serwera zbierającego dane}

\section{Podsumowanie}

\begin{thebibliography}{9}

\bibitem{dz}
  Hałas w szkołach: Sprawdziliśmy. Jest głośniej niż fabryce, Dziennik Zachodni, 2012-10-04. Dostępny w internecie: \url{http://www.dziennikzachodni.pl/artykul/659865,halas-w-szkolach-sprawdzilismy-jest-glosniej-niz-fabryce-test-dz,id,t.html}

\bibitem{wikihalas}
  Hałas. Wikipedia: wolna encyklopedia, 2017-01-14 16:00. Dostępny w internecie: \url{https://pl.wikipedia.org/w/index.php?title=Ha%C5%82as&oldid=48179951}
  
\end{thebibliography}

\end{document}